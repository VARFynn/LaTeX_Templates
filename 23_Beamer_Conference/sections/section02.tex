\section{Empirical Framework} 
%----------------------------------------------------------------------------------------

%----------------------------------------------------------------------------------------
\subsection{Heterogeneity}
\begin{frame}{There exists quite some Heterogeneity in Effects}
\label{Slide12}
\begin{itemize}

\item \textbf{Geography: \hfill  \hyperlink{Slide12}{\beamerbutton{Table}}} 
\begin{itemize}
\item Higher Application effects in rural areas  vs. urban (48.2 vs. 29.3 pp). 
\item Pass-trough to enrollment reduces gap in enrollment rates by one half ($\Delta$ 14 vs. $\Delta$ 7.6 pp).
\end{itemize}
\vfill \pause
\item \textbf{Race: \hfill \hyperlink{Slide12}{\beamerbutton{Table}}} 
\begin{itemize}
\item Application effect dominates for White (44.5 pp vs. 22--29 pp; \textbf{no within region effects}). 
\item Weaker Pass-through for White (Appl.: 44.5 pp $\rightarrow$ Enroll.: 15.7 pp) and stronger for Asian (Appl.: 23.2 pp $\rightarrow$ Enroll.: 13.9 pp)

\end{itemize}
\vfill \pause

\item \textbf{Economic Status:} No stat. sign differences. Weak effect in higher Betas visible. \hfill \hyperlink{Slide12}{\beamerbutton{Table}}

\vfill \pause

\item \textbf{Gender:} Similarish Application Effects (42.4 vs. 40.3 pp) with slightly higher pass-through for Females (Enroll.: 16.4 vs. 12.1 pp).  \hfill \hyperlink{Slide12}{\beamerbutton{Table}}

\vfill \pause

\item \textbf{Isolated Students:} Effect larger for more isolated students. \hfill \hyperlink{Slide12}{\beamerbutton{Table}}


\end{itemize}
\end{frame}
%----------------------------------------------------------------------------------------

