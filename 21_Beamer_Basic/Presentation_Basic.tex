%%%%%%%%%%%%%%%%%%%%%%%%%%%%%%%%%%%%%%%%%%%%%%%%%%%%%%%%%%%%%%%%%%%%%%%%%%%%%%%%%%%%%%%
%%%%%%%%%%%%%%%%%%%%%%%%%%%%%%%%%%%%%%%%%%%%%%%%%%%%%%%%%%%%%%%%%%%%%%%%%%%%%%%%%%%%%%%%
% Beamer Template - Version: Quick and Beautiful 
% © Fynn Lohre 
%
% The following template is designed for a quick presentation in standard style.
%
% In case of any questions - feel free to reach out: VARFynn@gmail.com
% If this template helps you, endorsements on LinkedIn or in persona are welcome.
%
%%%%%%%%%%%%%%%%%%%%%%%%%%%%%%%%%%%%%%%%%%%%%%%%%%%%%%%%%%%%%%%%%%%%%%%%%%%%%%%%%%%%%%%%

%---------------------------------------------------------------------------------------

%%%%%%%%%%%%%%%%%%%%%%%%%%%%%%%%%%%%%%%%%%%%%%%%%%%%%%%%%%%%%%%%%%%%%%%%%%%%%%%%%%%%%%%%
%
% 1.  Preamble
%
% This defines the properties of the document.
% Includes: Font Size, Spacing & other "Style Elements".
%
%%%%%%%%%%%%%%%%%%%%%%%%%%%%%%%%%%%%%%%%%%%%%%%%%%%%%%%%%%%%%%%%%%%%%%%%%%%%%%%%%%%%%%%%

%% 1.1 Basic Packages

\documentclass[11pt,aspectratio=169] {beamer}
\usepackage[utf8]{inputenc}
\usepackage{babel}
\usepackage{amsmath}
\usepackage{amsfonts}
\usepackage{amssymb}
\usepackage{mathtools}
\usepackage{tikz}
\setbeamercovered{transparent} 
%\setbeamertemplate{navigation symbols}{} 
\setbeamertemplate{caption}[numbered]
\setbeamertemplate{theorems}[numbered] 

%% 1.2 Coloring Possibilities
% I usually switch between those two themes, sometimes combined with seahorse.
\usetheme{CambridgeUS}
%\usetheme{PaloAlto}
%\usecolortheme{seahorse}

% This changes the blocks to a different style (I only use it for the CambridgeUS red style).
\setbeamercolor{block title}{bg=red!60!black, fg=white} 
\setbeamercolor{block body}{bg=white, fg=black} 

% This changes the style of the TOC Bullets and items from the standard blue to red like rectangles - However,  it is optional,  and I only use it sometimes.
\useinnertheme{rectangles}
\setbeamercolor{section number projected}{bg=red!60!black}  				% If you don't want it u could put bg - background to white - and fg - foreground - to black)
\setbeamercolor{subsection number projected}{bg=red!60!black}
\setbeamercolor{item}{fg=red!60!black}

%% 1.3 Pre Info for Titlepage

\title[Short Title for Fooder]{Titel}							% Usage of Short Names ist optional,  just remove [] if not needed
\author[Short Name for Fooder]{John Doe}
\institute[Short Name for Institute]{Institute}

%---------------------------------------------------------------------------------------

%%%%%%%%%%%%%%%%%%%%%%%%%%%%%%%%%%%%%%%%%%%%%%%%%%%%%%%%%%%%%%%%%%%%%%%%%%%%%%%%%%%%%%%%
%
% 2. Slides
%
%%%%%%%%%%%%%%%%%%%%%%%%%%%%%%%%%%%%%%%%%%%%%%%%%%%%%%%%%%%%%%%%%%%%%%%%%%%%%%%%%%%%%%%%

\begin{document}

% Title Page
\begin{frame}
\titlepage
\end{frame}

% TOC 
\begin{frame}{Table of Contents}
\tableofcontents
\end{frame}
\section{Type Section I  Here}
\subsection{Type Subsection I Here}

\begin{frame}{Table of Contents II Version}
 \tableofcontents
    [
        currentsection,
        currentsubsection,
        subsectionstyle=show/shaded/hide
    ]
\end{frame}

% Equations slide
\begin{frame}{Equations}
\[
E=mc^2
\]
\begin{equation}
\hat{\beta} = 
\end{equation}
\end{frame}

% Double-Sided slide
\begin{frame}[allowframebreaks]{Double-Sided Slide}
This slide is split into multiple frames.
\framebreak
Second part of the slide.
\end{frame}

% Block slide
\begin{frame}{Block}
\begin{block}{Block Title}
This is a block.
\end{block}
\begin{alertblock}{Alert Block Title}
This is an alert block.
\end{alertblock}
\begin{exampleblock}{Example Block Title}
This is an example block.
\end{exampleblock}
\end{frame}

% Theorem slide
\begin{frame}
\begin{Theorem}
XYZ 
\end{Theorem}
\end{frame}

% Overlay slide
\begin{frame}{Overlay}
\begin{itemize}
\item<1-> Item 1
\item<2-> Item 2
\item<3-> Item 3
\end{itemize}
\end{frame}

% Image and Text slide
\begin{frame}{Image and Text}
\begin{columns}
\begin{column}{0.4 \textwidth}
\begin{figure}
\includegraphics[width=\textwidth]{example-image}
\caption{Caption}
\end{figure}
\end{column}
\begin{column}{0.4\textwidth}
Some text.
\end{column}
\end{columns}
\end{frame}

% Bullet Points slide
\begin{frame}{Bullet Points \& Enumerate}
\begin{itemize}
\item Bullet point 1
\item Bullet point 2
\item Bullet point 3
\end{itemize}
\begin{enumerate}
\item Bullet point 1
\item Bullet point 2
\item Bullet point 3
\end{enumerate}
\end{frame}

% Table slide
\begin{frame}{Table}
\begin{tabular}{|c|c|}
\hline
A & B \\
\hline
1 & 2 \\
3 & 4 \\
\hline
\end{tabular}
\end{frame}

% Top Centered Slide
\begin{frame}[t]{Top Centered Slide}
This slide is centered at the top.
\end{frame}

%E pochal Slide
\begin{frame}{Epochal Slide}
\centering
\textbf{\Huge Epochal Event}
\end{frame}

% Quote Slide
\begin{frame}{Quote Slide}
\begin{quote}
This is a quote.
\end{quote}
\end{frame}

% Code Slide
\begin{frame}[fragile]{Code Slide}
\begin{verbatim}
for i in range(5):
    print(i)
\end{verbatim}
\end{frame}
\section{Type Section II Here}

% Image Grid Slide
\begin{frame}
\frametitle{Image Grid Slide}
\begin{figure}
\includegraphics[width=0.3\textwidth]{example-image-a}
\includegraphics[width=0.3\textwidth]{example-image-b}
\includegraphics[width=0.3\textwidth]{example-image-c}
\caption{Images in a grid}
\end{figure}
\end{frame}

%Blank Frame
\bgroup
\setbeamercolor{background canvas}{bg=black}
\begin{frame}[plain]{}
\end{frame}
\egroup

\section{Backup}
% Two Columns Slide with Divider
\begin{frame}[t]
\frametitle{Two Columns Slide with Divider}
\begin{columns}[T]
\column{0.46\textwidth}
\underline{\textbf{Column 1 Subhead}}
\begin{itemize}
\item Text
\item Text
\end{itemize}

\column{0.02\textwidth}
\centering
\rule{0.5pt}{4cm} % Vertical divider

\column{0.46\textwidth}
\underline{\textbf{Column 2 Subhead}}
\begin{itemize}
\item Text
\item Text
\end{itemize}
\end{columns}
\vfill
\begin{block}{Takeaway Box}
Text
\end{block}
\vfill
\end{frame}

\end{document}