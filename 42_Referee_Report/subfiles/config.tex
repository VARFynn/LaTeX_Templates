% Essential packages for professional formatting incl. everything needed to display equations (if needed). 
%---------------------------------------------------------------------------------------

\usepackage[utf8]{inputenc}                                     % Ensures UTF-8 encoding
\usepackage[T1]{fontenc}                                        % Improved font encoding for better output
\usepackage[english]{babel}                                     % Language support (set to English, adjust as needed)
\usepackage[margin=1in]{geometry}                               % Sets 1-inch margins
\usepackage{amsmath, amssymb, amsthm}                           % Core math packages
\usepackage{mathtools}                                          % Enhanced math features
\usepackage{bm}                                                 % Bold math symbols
\usepackage{physics}                                            % Convenient commands for physics and derivatives
\usepackage{fancyhdr}                                           % For header and footer customization
\usepackage{datetime}                                           % To display the date in various formats
\usepackage{enumitem}                                           % For better control over lists
\usepackage[colorlinks=true,  
            linkcolor=blue, 
            urlcolor=blue, 
            citecolor=blue]{hyperref}                           % For clickable links
\usepackage{xcolor}                                             % Colored text for highlighting

% Configure page style
\pagestyle{fancy}                                               % Use fancyhdr package for custom footer
\fancyhf{}                                                      % Clear default header and footer
\renewcommand{\headrulewidth}{0pt}                              % Remove header rule
\fancyfoot[C]{\thepage}                                         % Centered page numbering in the footer

% Math environments
\theoremstyle{definition} % Style for definitions
\newtheorem{definition}{Definition}[section]

\theoremstyle{plain} % Style for theorems
\newtheorem{theorem}{Theorem}[section]

\theoremstyle{remark} % Style for remarks
\newtheorem*{remark}{Remark}

% Custom Items
\newcommand{\mitem}[2]{\item \textbf{#1:} {#2}}
