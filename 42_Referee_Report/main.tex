%%%%%%%%%%%%%%%%%%%%%%%%%%%%%%%%%%%%%%%%%%%%%%%%%%%%%%%%%%%%%%%%%%%%%%%%%%%%%%%%%%%%%%%
%%%%%%%%%%%%%%%%%%%%%%%%%%%%%%%%%%%%%%%%%%%%%%%%%%%%%%%%%%%%%%%%%%%%%%%%%%%%%%%%%%%%%%%%
% Referee Report with Letter to the editor
% © Fynn Lohre 
%
% The following template is designed for a quick setup to minimize time usage.
% In theory, you should just use the main file.
% However, if you need to customaize something look into subfiles/.
%
% In case of any questions - feel free to reach out: VARFynn@gmail.com
% If this template helps you, endorsements on LinkedIn or in persona are welcome.
%
%%%%%%%%%%%%%%%%%%%%%%%%%%%%%%%%%%%%%%%%%%%%%%%%%%%%%%%%%%%%%%%%%%%%%%%%%%%%%%%%%%%%%%%%

%---------------------------------------------------------------------------------------
% PRE
\documentclass[11pt]{article}

%---------------------------------------------------------------------------------------
% ADD FURTHER PACKAGES HERE (If needed, otherwise go to subfiles/config.tex


% Essential packages for professional formatting incl. everything needed to display equations (if needed). 
%---------------------------------------------------------------------------------------

\usepackage[utf8]{inputenc}                                     % Ensures UTF-8 encoding
\usepackage[T1]{fontenc}                                        % Improved font encoding for better output
\usepackage[english]{babel}                                     % Language support (set to English, adjust as needed)
\usepackage[margin=1in]{geometry}                               % Sets 1-inch margins
\usepackage{amsmath, amssymb, amsthm}                           % Core math packages
\usepackage{mathtools}                                          % Enhanced math features
\usepackage{bm}                                                 % Bold math symbols
\usepackage{physics}                                            % Convenient commands for physics and derivatives
\usepackage{fancyhdr}                                           % For header and footer customization
\usepackage{datetime}                                           % To display the date in various formats
\usepackage{enumitem}                                           % For better control over lists
\usepackage[colorlinks=true,  
            linkcolor=blue, 
            urlcolor=blue, 
            citecolor=blue]{hyperref}                           % For clickable links
\usepackage{xcolor}                                             % Colored text for highlighting

% Configure page style
\pagestyle{fancy}                                               % Use fancyhdr package for custom footer
\fancyhf{}                                                      % Clear default header and footer
\renewcommand{\headrulewidth}{0pt}                              % Remove header rule
\fancyfoot[C]{\thepage}                                         % Centered page numbering in the footer

% Math environments
\theoremstyle{definition} % Style for definitions
\newtheorem{definition}{Definition}[section]

\theoremstyle{plain} % Style for theorems
\newtheorem{theorem}{Theorem}[section]

\theoremstyle{remark} % Style for remarks
\newtheorem*{remark}{Remark}

% Custom Items
\newcommand{\mitem}[2]{\item \textbf{#1:} {#2}}


%---------------------------------------------------------------------------------------
% STEP 1: Customize Basics
\newcommand{\authorName}{Your Name}                                                      % Author's name
\newcommand{\schoolName}{Your School}                                                    % School/affiliation
\newcommand{\address}{Your Address}                                                      % Address
\newcommand{\editorName}{Editor Name}                                                    % Editor's name
\newcommand{\journalName}{Journal Name}                                                  % Journal name
\newcommand{\editorAddress}{Editor Address}                                              % Address of the journal editor
\newcommand{\manuscriptID}{1234 }                                                        % Manuscript ID
\newcommand{\manuscriptTitle}{This is the Title}                                         % Manuscript title
\newcommand{\reviewDate}{\today}                                                         % Date of review (default is today)


%---------------------------------------------------------------------------------------
% STEP 2: Add Core Parts in Letter to the EDITOR 
% For a more in depth change, go to letter_editor.tex

 % Recommendation for the manuscript (Acceptance/ R&R / Reject)
\newcommand{\recommendation}{RECOMMENDATION}                   
           
% Assesment of the manuscript (Quick Summary of your impression)
\newcommand{\assessment}{ASSESSMENT}      

% Brief summary of the manuscript (Summary of core parts for the editor)
\newcommand{\summaryPart}{SUMMARY PART}              

% Assesment of Limitations and potential flaws (List the core flaws of the paper; in the end the editor has to decide)
\newcommand{\limitations}{LIMITATIONS / FLAWS}                  

% Core Part Letter to the Editor.
% If you want to customaize it, do it here.

%---------------------------------------------------------------------------------------

% Document begins
\begin{document}

% Cover Letter Section
\begin{flushright}
\noindent \authorName \\ 
\schoolName \\
\address \\ 
\end{flushright}

\bigskip

% Address to the Editor
\noindent \editorName \\ 
\journalName \\ 
\editorAddress

\bigskip
\begin{flushright}
\noindent \reviewDate % Inserts the date
\end{flushright}

\bigskip
% Subject Line
\noindent Re: Manuscript ID \manuscriptID -- 
Title: ``\manuscriptTitle''

\bigskip

% Greeting
\noindent Dear \editorName,

\bigskip

% Opening Statement
\noindent I have completed my review of the above-mentioned manuscript. Please find my detailed referee report attached below. My assessment is \assessment. \textbf{Hence, my recommendation is a \recommendation.}

\bigskip

% Summary Section
\noindent In summary, this manuscript represents \summaryPart.

\bigskip

% Limitations or Flaws Section
\noindent Although these findings \limitations.

\bigskip

% Closing
\noindent Best,\\[0.3cm]
\authorName % Inserts your name

%---------------------------------------------------------------------------------------
% STEP 3: Summary of the Paper or the Author
\newcommand{\summaryauthor}{This is the summary of the paper and some initial comments.}   

%---------------------------------------------------------------------------------------
% STEP 4: Major Items
% just use \mitem{Summary word(s)}{Text}
\newcommand{\majoritem}{
\mitem{Problem Identification}{This is a major item, whereas a problem in the identification exists}
\mitem{Problem Identification}{This is a major item, whereas a problem in the identification exists}
\mitem{Problem Identification}{This is a major item, whereas a problem in the identification exists}
\mitem{Problem Identification}{This is a major item, whereas a problem in the identification exists}
}

%---------------------------------------------------------------------------------------
% STEP 5: Minor Items
% just use \mitem{Summary word(s)}{Text}
\newcommand{\minoritem}{
\mitem{Problem Identification}{This is a minor item, whereas a problem in the identification exists}
\mitem{Problem Identification}{This is a minor item, whereas a problem in the identification exists}
\mitem{Problem Identification}{This is a minor item, whereas a problem in the identification exists}
\mitem{Problem Identification}{This is a minor item, whereas a problem in the identification exists}
}

%---------------------------------------------------------------------------------------
% STEP 5: Final Comments
\newcommand{\finalcomment}{
This is the final comment. This can be typos or misc. things. 
}
% Core Part Letter to Referee Report.
% If you want to customaize it, do it here.

%---------------------------------------------------------------------------------------

\newpage
\pagenumbering{roman}
% Referee Report Section
\begin{center}
\Large\textbf{\textsc{Referee Report}}

\end{center}
\bigskip

\noindent \large\textbf{\textsc{initial comments}}

\vspace{0.3cm}
\noindent \summaryauthor



% Define a custom enumerate style for major items
\setlist[enumerate,1]{label=A\arabic*, ref=A\arabic*}          % Level 1: A1, A2, A3, ...
\setlist[enumerate,2]{label=A\arabic*.\arabic*, ref=A\arabic*.\arabic*}  % Level 2: A1.1, A1.2, ...
\setlist[enumerate,3]{label=A\arabic*.\arabic*.\arabic*, ref=A\arabic*.\arabic*.\arabic*} % Level 3: A1.1.1, A1.1.2, ...
\setlist[enumerate,4]{label=A\arabic*.\arabic*.\arabic*.\arabic*, ref=A\arabic*.\arabic*.\arabic*.\arabic*} % Level 4: A1.1.1.1, A1.1.1.2, ...

\bigskip
\noindent \large\textbf{\textsc{major items}}
\begin{enumerate}
\majoritem
\end{enumerate}

\setlist[enumerate,1]{label=B\arabic*, ref=B\arabic*}          % Level 1: B1, ... 
\setlist[enumerate,2]{label=B\arabic*.\arabic*, ref=B\arabic*.\arabic*}  % Level 2: B1.1, ....
\setlist[enumerate,3]{label=B\arabic*.\arabic*.\arabic*, ref=B\arabic*.\arabic*.\arabic*} % Level 3: B1.1.1, ....
\setlist[enumerate,4]{label=B\arabic*.\arabic*.\arabic*.\arabic*, ref=B\arabic*.\arabic*.\arabic*.\arabic*} % Level 4: B1.1.1.1, ...

\bigskip
\noindent \large\textbf{\textsc{minor items}}
\begin{enumerate}
\minoritem
\end{enumerate}

\bigskip
\noindent \large\textbf{\textsc{final comments}}

\vspace{0.3cm}
\noindent \finalcomment

\end{document}
